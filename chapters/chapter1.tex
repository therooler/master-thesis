\chapter{Introduction}

The field of quantum machine learning is a rapidly developing area at the intersection of quantum physics and machine learning. To make progress in this field we require a firm understanding of the basic ingredients of quantum mechanics, which are established in the first chapter. To understand some of the numerical difficulties one encounters when dealing with quantum systems, we will discuss an implementation of a Path Integral quantum annealing scheme, reproducing some of the work done in the early 2000s and outlining the discussion in the decades following it. In the final part we will consider the quantum perceptron, a novel algorithm with a generalization of the classical log-likelihood function.  In each section a literature review is included.

All code was written in Python 3, with exception of the Path Integral Quantum Monte Carlo schedule, which was implemented in Cython. 
